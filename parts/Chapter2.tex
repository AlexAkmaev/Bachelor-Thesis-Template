\section{Обзор существующих решений}
\label{sec:Chapter2} \index{Chapter2}

\subsection{Формализация системы типов на основе Featherweight Java}

В данной статье вводятся типы объединения для объектно-ориентированных языков, основанных на статически
типизированных классах [4].
Целью работы является реализация эффективного использования разнородных коллекций и группировка независимо определеннх
классов с аналогичными интерфейсами, нивелируя сложность подобных возможностей в Java.
Реализуются объединения, которые могут быть представлены группой классов путем формирования их супертипа после
определения этих классов.
Тип объединения позволяет получить доступ к какому-либо полю или методу, играя роль интерфейса, состоящего из общих
свойств классов в него входящих.
Также в этой статье формализуется ядро системы типов поверх Featherweight Java и доказывается, что система типов надежна.
Хоть и ожидается, что она полезна сама по себе, механизм прямого доступа к элементам можно вполне подвергнуть критике
за то, что он в основном зависит от равенства имен свойств классов, что может вполне оказаться случайным совпадением.
Стоит отметить следующие способы реализации некоторых механизмов:

\begin{itemize}[left=2em]
    \item Объединение \hl{A|B} преобразуется в общий супертип A и B или просто в Object.
    \item Case и прямой доступ к элементам выражаются в терминах instanceOf и downcasts.
\end{itemize}

Рассмотрим подходы к реализации объединений, которые представлены в этой статье.
Первое утверждение заключается в том, что \hl{A|B} является супертипом как A, так и B\@.
Из этого следует, что для классов A и B допускается приведенное ниже присвоение:

\begin{lstlisting}
    A|B un = new A();
\end{lstlisting}

Более того, \hl{A|B} является наименьшим супертипом среди супертипов A и B в том смысле, что любой общий супертип A и
B также является супертипом \hl{A|B}.
Таким образом, такое присваивание также разрешено:

\begin{lstlisting}
    C x = un;
\end{lstlisting}

Здесь предполагается, что классы A и B наследуются от класса C.
Главным образом объединения можно интерпретировать как тип-множество, но не как супертип всех его составляющих.
Это ознаает, что тип \hl{A|B} включает в себя только экземпляры A или B, и ничего больше.
В то время как другие супертипы могут включать экземпляры, принадлежащие к классам, отличным от A и B\@.
Также обращается внимание на то, что отношение подтипов не является антисимметричным, как в обычных
объектно-ориентированных языках.
Существуют два синтаксически различных типа, которые являются подтипами друг друга.
Например, \hl{A|B} и \hl{B|A} синтаксически различны и являются подтипами друг друга.
В статье предоставлены два вида операций с типами объединения: case-анализ и доступ к свойствам классов.
Case-анализ - это конструкция условий, которая разветвляется в соответствии с
классом значения тестируемого выражения, известного во время выполнения.
Например, в данном ветвлении

\begin{lstlisting}
  case un of (A x) { x.foo(); }
           | (B y) { y.bar(42); y.foo(); }
\end{lstlisting}

вызывается метод foo(), если значение un является экземпляром A (или одного из его подклассов), или методы bar() и foo(),
если фактический класс является экземпляром класса B (или одному из его подклассов).
Здесь x и y имеют конкретный тип соотвествующего класса, но статически их тип вычисляется как \hl{A|B}.
В этом смысле такую конструкцию можно рассматривать как комбинацию динамического тестирования типов
во время исполнения (instanceOf) и приведения типов.
Таким образом, код можно переписать следующим образом:

\begin{lstlisting}
    if (un instanceOf A) { A x = (A)un; x.foo(); }
    else { B y = (B)un; y.bar(42); y.foo(); }
\end{lstlisting}

Одним из преимуществ использования этой конструкции является то, что система типов может проверять полноту
условий ветвления на соответствие тестируемому выражению.
Фактически, требуется, чтобы тип тестируемого выражения был подтипом объединения типов, представленных в ветвях (в
приведенном выше примере, A и B).
Это требование гарантирует, что будет исполнена любая ветвь и ее выполнение завершится успешно.
С другой стороны, стандартные системы типов не гарантируют успешность приведения типов во второй ветке.
Прямой доступ к элементам позволяет напрямую обращаться к полям объединений, если их компоненты содержат поля с
одинаковыми именами.
В качестве примера рассматриваются следующие определения A и B:

\begin{lstlisting}
    class A extends C {
        Int fld1;
        Int fld2;
        void foo(Int x) { ... }
    }
    class B extends C {
        Int fld1;
        Byte fld2;
        void foo(Int x) { ... }
    }
\end{lstlisting}

При таком определении классов, возможен прямой доступ к полю fld1 переменной un, имеющей тип \hl{A|B}:

\begin{lstlisting}
    Int i = un.fld1;
\end{lstlisting}

Более того, даже если типы полей различаются, допускается чтение из поля с общим именем:

\begin{lstlisting}
    Int|Byte i = un.fld2;
\end{lstlisting}

В таком случае возвращается объединение типов полей из разных классов.
Однако в нашей работе такой доступ к полям не удовлетворяет спецификации и ограничениям языка.

В свою очередь к вызовам методов предъявляются более строгие требования, поскольку методы с одинаковыми именами могут
иметь разные сигнатуры.
Вызов метода разрешается только в том случае, если имена, количество аргументов и соответствующие
типы аргументов полностью совпадают.
Следовательно, такой код отработает корректно:

\begin{lstlisting}
    un.foo(new Int(42));
\end{lstlisting}

Возвращаемые типы у методов могут отличаться, если эти типы являются объектными (не void), также, как при доступе к полю.
В дополнение, в статье рассматривается ослабление требований к типам в том смысле, что тип аргумент метода является подходящим,
если каждый фактический тип аргумента является подтипом обоих соответствующих формальных типов аргументов.
Однако такое ослабление доставляет массу проблем при перегрузке методов.
Таким образом, прямой доступ к свойствам классов, входящих в объединение, обеспечивает гораздо более лаконичный способ
вызова методов или модификации полей, чем использование case-анализа, когда компоненты объединения
содержат элементы с общими именами.
С помощью этого механизма тип объединения можно рассматривать как своего рода тип интерфейса, который “вычленяет” общие
элементы из его компонент.
Ожидается, что этот механизм будет полезен когда объединяются независимо определенные классы со схожими функциональными
возможностями.
Например, A и B могли быть определены отдельно от класса C.
В таком случае общий суперкласс A и B мог бы быть разве что только Object.
Даже в таком случае экземпляры этих двух классов могут обрабатываться совместно с помощью объединения \hl{A|B}, и, более того,
запрещается смешивать с ними экземпляры других классов (если только они не являются подклассами A или B).
Высказывается мнение, что интерфейсы в стиле Java и типы объединений являются скорее дополняющими друг друга
механизмами, нежели конфликтующими.

С одной стороны, явно объявленные интерфейсы полезны для абстрагирования от реализаций классов, а также для улучшения
документации, поскольку интерфейс предоставляет не только сигнатуры методов, но и
более семантические (или поведенческие) концепции реализующих его классов.
Например, “метод sort() действительно должен выполнять сортировку” (если такая функция не предусмотрена
языком программирования по умолчанию).
С другой стороны, объединения более полезны для предоставления апостериорных интерфейсов
для legacy или сторонних классов, над которыми программисты не всегда имеют контроль.

Все идеи, упомянутые выше, могут быть отлично использованы и в нашей работе,
однако эффективная реализация прямого доступа к элементам без громоздких ветвлений не достаточно хорошо
исследована в данной статье.

\subsection{Сочетание статической и динамической типизации на примере StaDyn}

StaDyn - это объектно-ориентированный язык программирования, основанный на C\# 3.0, который поддерживает как динамическую,
так и статическую типизацию [1].
Хотя текущая реализация StaDyn поддерживает большую часть функциональности C\#, ее минимальное ядро сосредоточено на
формализации того, как включить динамическую и статическую типизацию в один и тот же язык программирования.
В ядре StaDyn ссылки на переменные могут быть установлены как статически (по умолчанию), так и динамические, что изменяет
способ проверку типов.

Ядро StaDyn собирает информацию о типах во время компиляции, чтобы статически осуществить проверку типов по
динамическим ссылкам.
Одним из способов это сделать в этой работе использовались типы объединения.
Тип объединения \hl{A1|A2} определяет обычное объединение множества значений, принадлежащих A1 и набор значений,
принадлежащих A2, представляющий наименьшую верхнюю границу значений A1 и A2.
Тип объединения содержит все возможные типы, которые может иметь ссылка.
Набор операций (например, добавление, доступ к полю, присвоение, вызов или индексация), которые могут быть применены к
типу объединения, определяется каждым типом, входящим в объединения.
То есть общее подмножество всех операций, которые разрешены для всех типов в объединении, составляют этот набор.
Типы объединений уже были включены в объектно-ориентированные языки, в системы типов, где они были явно описаны
[2] или выведены из неявно типизированных ссылок [3].
В этой статье взяли другие правила подтипирования, добавив новое правило динамической типизации.
Если тип объединения является статическим, то набор операций определяется так, как было сказано ранее.
В том случае, если ссылка на объединение динамическая, проверка типов является менее строгой.
На практике это означает, что операция становится примменима к обхединению, если она применима хотя бы к одному
из типов, входящих в него.
Если операция не может быть применена к какому-либо типу, будет сгенерирована ошибка типа, даже если ссылка является
динамической.
В этой новой интерпретации тип объединения \hl{A1|A2} может представлять собой разную сущность в среде выполнения.
В случае статической ссылки он является наименьшей верхней границей значений A1 и A2, а в случае динамической ссылки
- каким-то из типов, входящих в объединение.

Проанализировав результаты данной статьи, можно заключить, что подход смешанной типизации позволяет языку программирования
StaDyn повышать производительность динамической типизации во время выполнения и гибкость статической типизации.
Его система типов выполняет вывод типов как из статических, так и из динамических неявных ссылок, чтобы улучшить
производительность во время выполнения и статически проверять динамические типы.
В то же время информация о типах, собранная компилятором, позволяет взаимодействовать обоим типам кода, используя
одну и ту же систему типов.
Оптимизация StaDyn основана на статистическом получении информации о типе динамических ссылок.
Наибольшая выгода достигается при выполнении тестов с динамической типизацией.
Однако в этой статье помимо всего прочего необходимо формализовать семантику ядра языка и доказать его
типобезопасность при использовании статических ссылок.


\subsection{Теоретико-множественные типы}

В этой статье описывается использование теоретико-множественных типов в языках программирования и
излагается их теория [5].
Теоретико-множественные типы включают типы объединения T1|T2, типы пересечения
T1\&T2 и типы отрицания !T\@.
В строгих языках имеет смысл интерпретировать тип как набор значений, которые имеют этот тип
(например, Bool интерпретируется как набор, содержащий значения true и false).
Таким образом, согласно этому предположению,

\begin{itemize}[left=2em]
    \item \hl{T1|T2} - это набор значений, которые относятся либо к типу T1, либо к типу Y2;
    \item \hl{T1\&T2} - это набор значений, которые относятся как к типу T1, так и к типу T2;
    \item \hl{!T} - это набор всех значений, которые не относятся к типу T\@.
\end{itemize}

Теоретико-множественные типы являются полиморфными, когда они включают типовые переменные.
Для того, чтобы дать представление о способы программирования, в котором используются использовать теоретико-множественные
типы и который описывается в этой статье, рассматривается классическая рекурсивная функция сглаживания, которая преобразует
произвольно вложенные списки в список их элементов.
В ML-подобном языке с сопоставлением с образцом это может быть определено так же просто, как

\begin{lstlisting}
    let rec flatten = function
        | [] -> []
        | h::t -> (flatten h)@(flatten t)
        | x -> [x]
\end{lstlisting}

Данный код можно интерпертировать следующим образом:

\begin{itemize}[left=2em]
    \item Функция flatten возвращает пустой список [], когда ее аргумент является пустым списком.
    \item Если аргумент является непустым списком, то она сглаживает начало h и конец t аргумента и возвращает объединение
результатов (обозначается символом @).
    \item Если аргумент не является списком (т.е., первые два пункта не выполняются), функция flatten возвращает список,
содержащий только этот аргумент.
\end{itemize}

Функция flatten полностью полиморфна: она может быть применена к любому аргументу и, если списки конечны, всегда
завершает работу.
Хотя семантику функции легко понять, присвоение ей простого и общего полиморфного типа противоречит всем существующим
языкам программирования [7], за единственным исключением: CDuce [6].
Это связано с тем, что CDuce - это язык, который использует полный набор теоретико-множественных связок типов, и
тут необходимо они все (объединение, пересечение и отрицание) для определения Tree(a), типа вложенных списков,
элементы которых относятся к типу a.

\begin{lstlisting}
    type Tree(a) = (a\List(Any)) | List(Tree(a))
\end{lstlisting}

В этом определении используются следующие обозначения:
\begin{itemize}[left=2em]
    \item Символ “|” означает объединение.
    \item Символ “\textbackslash” обозначает разность, то есть пересечение с отрицанием: \hl{T1\textbackslash T2 = T1\&!T2}.
    \item List(T) - это список элементов типа T\@.
    \item Any - это тип любых значений, так что List(Any) - это тип любого списка.
\end{itemize}
Другими словами, \hl{Tree(a)} - это тип вложенных списков, листья которых, то есть элементы, не являющиеся списками, имеют тип a.
Таким образом, это либо лист, либо список Tree(a).
Тогда достаточно просто указать в аннотации flatten правильный тип.

\begin{lstlisting}
    let rec flatten: Tree(a)!List(a) = function ...
\end{lstlisting}

Важным моментом является то, что независимо от типа аргумента flatten, выражение всегда хорошо типизировано.
Если аргумент не является списком, то создается экземпляр a в соответствии с типом аргумента.
Если это список, то это также вложенный список, и создается экземпляр a с объединением типов элементов, не входящих в
этот вложенный список.
Другими словами, сглаживание может быть применено к выражениям любого типа, и тип, выводимый для такого применения, -
это \hl{List(T)}, где тип T является объединением типов всех конечных элементов аргумента, причем аргумент, не относящийся к
списку, сам по себе является конечным элементом.
Например, статически выведенный тип выражения

\begin{lstlisting}
    flatten [3 "r" [4 [true 5]] ["quo" [[false] "stop"]]]
\end{lstlisting}
будет, соответственно, типом \hl{List(Int|Bool|String)}.

Одной из ключевых особенностей полиморфных теоретико-множественных типов, делающей их универсальными, является то, что
они включают в себя все три основные формы полиморфизма:

\begin{itemize}[left=2em]
    \item Параметрический полиморфизм: описывает код, который может работать с любым типом.
    По сути это свойство семантики системы типов позволяет обрабатывать значения разных типов одинаковым образом,
    то есть выполнять один и тот же код для данных различных типов.
    В этой статье рассматривается только так называемый полиморфизм второго класса (в смысле [8]), когда количественная
    оценка переменной не может отображаться ниже конструкторов типов или связок типов;
    \item Специальный (ad-hoc) полиморфизм: позволяет коду работать с несколькими типами, возможно, с разным поведением
    в каждом случае, как при перегрузке функций.
    Он обеспечивает единый интерфейс к разнообразному коду для работы с различными типами,
    которые могут быть несовместимыми, но допустимыми в данном контексте.
    Например, он позволяет иметь одну и ту же реализацию для разных типов в случае с оператором \hl{+}
    (сложение Int или сложение String);
    \item Полиморфизм подтипов: создает иерархию более или менее точных типов для одного и того же кода, позволяя
    использовать его везде, где ожидается любой из этих типов.
\end{itemize}

Также в частном порядке автором были рассмотрены полиморфные теоретико-множественные типы в более общей постановке,
показывая, как эти типы позволяют эффективно вводить некоторые функции и идиомы языков программирования.
Это было проиллюстрировано на примере условных ветвлений с применением объединений.
В языке с типами объединений мы можем вводить точные условные выражения, которые возвращают результаты разных типов.
Например,

\begin{lstlisting}
    if e then 3 else true
\end{lstlisting}
имеет тип \hl{Int|Bool} (при условии, что e имеет тип Bool).
Без типов объединения он мог бы иметь приблизительный тип, например, наименьший супертип тип всех типов исходящих из
всех веток, или попросту быть неправильно типизированным.
Типы объединений могут также быть использованы для структур, подобных спискам, для смешения разные типов на примере
ранее продемонстрированного выражения flatten, которое вернуло список типа \hl{List(Int|Bool|String)}.

Это делает объединения незаменимыми для разработки систем типов для существующих нетипизированных
языков: примером может служить их включение в Typed Racket [9], которое позволяет
автоматически добавлять аннотации к статически проверяемым типам на диалекте Scheme и в
TypeScript [10], и Flow [11], которые расширяют JavaScript за счет статической проверки типов.

В своей статье автор попытался рассмотреть многочисленные преимущества и способы использования теоретико-множественных
типов в программировании.
Теоретико-множественные типы иногда являются единственным способом ввода некоторых конкретных функций, иногда таким же
простым, как функция сглаживания, описанная в начале.
Это происходит потому, что теоретико-множественные типы предоставляют подходящий язык для описания многих нетрадиционных,
но нередких шаблонов программирования.
Это подтверждается тем фактом, что потребность в теоретико-множественных типах естественным образом возникает при
попытке соответствовать системам типов в динамических языках: объединение и отрицание становятся необходимыми для
понимания природы ветвления и сопоставления с образцом, пересечения часто являются единственным способом описания
полиморфного использования некоторых функций, в определении которых отсутствует единообразие, требуемое параметрическим
полиморфизмом.
Развитие таких языков, как Flow, TypeScript и Typed Racket, является хорошим доказательством этого современного тренда.

Автор также показал, что даже при использовании теории множеств типы не всегда доступны программисту, они часто
присутствуют на метауровне, поскольку предоставляют базовые инструменты для точного ввода некоторых программных конструкций,
таких как варианты типов и сопоставление с образцом.
Теоретико-множественные типы предоставляют мощный теоретический инструментарий для изучения, понимания и формализации
существующих дисциплин, связанных с типами.
Автор продемонстрировал это на примере постепенных типов, которые, благодаря теоретико-множественным типам, могут быть
поняты как интервалы статических типов, аналогия, которую можно использовать для переосмысления их теории и их
практической реализации.
Этот обзор, безусловно, неполный.
Например, автор почти не говорил о типах XML и XML-программировании, хотя они послужили первой мотивацией для разработки
теории семантического подтипирования, а также для разработки и внедрения таких языков программирования, как XDuce и CDuce.
Автор также не упоминал, как обращаться с функциями, которые распространены в современных языках программирования,
такими как использование абстрактных типов, интеграция которых со структурными подтипами и полиморфизмом может привести
к появлению побочных эффектов.

С формальной точки зрения пока не удалось определить уникальный формализм, который сочетал бы неявно и явно
типизированные функции, реконструкцию типов пересечений и расширенное использование типизации вхождений.
Но исследователи не так уж далеки от этого.
С практической точки зрения требуется еще больше работы.
Параметрический полиморфизм с теоретико-множественными типами подразумевает генерацию и разрешение ограничений.
У этого есть несколько недостатков.
Во-первых, из-за наличия объединений и подтипов, решение задач по ограничению является потенциальным источником
вычислительного взрыва, с которым пока еще не очень хорошо справляются.
Во-вторых, устранение ограничений затрудняет генерацию информативных сообщений об ошибках в случае нарушения работы
программы, а также печать выведенных типов в форме, легко понятной программисту.

\subsection{Анализ существующих подходов}

В данном разделе были представлены различные подходы к типизации в объектно-ориентированных языках программирования.
Описание начинается со статьи о формализации системы типов на основе Featherweight Java, в которой вводятся типы
объединения для управления гетерогенными коллекциями и группировки независимо определенных классов с аналогичными
интерфейсами.
Это позволяет обеспечить эффективное использование гетерогенных коллекций и уменьшить сложность подобных возможностей
в языке Java.
Далее обзор переходит к рассмотрению языка программирования StaDyn, который поддерживает как динамическую, так и
статическую типизацию.
Здесь описывается смешанная типизация, где типы объединения играют важную роль.
Этот подход позволяет языку StaDyn повысить производительность динамической типизации во время выполнения и
гибкость статической типизации.
В заключительной части рассматривается использование теоретико-множественных типов в языках программирования.
Здесь описывается теория таких типов, как объединения, пересечения и отрицания.
Данная статья приводит примеры использования теоретико-множественных типов, включая рекурсивную функцию сглаживания
списков.
Она также подчеркивает важность использования таких типов для эффективной типизации в существующих и разрабатываемых
языках программирования.

В целом, были рассмотрены различные методы и концепции типизации, которые могут быть применены для повышения гибкости,
производительности и безопасности в объектно-ориентированных языках программирования.
Концептуально, подход к реализации объединений довольно схож в разных исследованиях, особенно это касается подтипирования.
С небольшими корректировками данные подходы будут использованы и в нашей работе.

\newpage
