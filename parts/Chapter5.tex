\section{Заключение}
\label{sec:Chapter5} \index{Chapter5}

Типы объединений решают несколько ключевых задач в статически типизированных языках, предоставляя гибкость
в представлении данных при сохранении безопасности типов.
Они улучшают возможность работы с различными формами данных, повышают ясность кода и обеспечивают более надежный и
безошибочный код благодаря проверкам на этапе компиляции.

Перечислим основные результаты нашей работы:
\begin{itemize}[left=2em]
    \item Проектирование и реализация union типов:
В анализаторе типов спроектирован и реализован класс для работы с объединениями, обеспечивающий безопасное и удобное
использование этих типов в коде.
Реализована возможность доступа к общим полям всех составляющих union типов, что позволяет писать более универсальный
и многократно используемый код.

    \item Нормализация типов:
Реализованы алгоритмы нормализации типов, входящих в объединение, что позволило устранить дублирующиеся и идентичные типы.

    \item Поддержка литералов и понижающие трансформации:
Реализована поддержка литералов в качестве составляющих объединения, что расширило возможности использования union типов.
Перенесены необходимые функциональные элементы в фазу понижающих трансформаций дерева, что, в том числе,
улучшило совместимость с TypeScript.

    \item Покрытие тестами и результаты оптимизации:
Процент успешно пройденных тестов составил более 90\%, что свидетельствует о высоком качестве реализации.
Оставшийся процент обусловлен небольшим количеством недостатков, которые постепенно будут устраняться.
    \item
Графики результатов оптимизации показали значительное уменьшение размера байт-кода и улучшение производительности программ.
Улучшение времени исполнения в интерпретаторе и JIT-е составило порядка трех раз.
\end{itemize}

Реализация поддержки union типов в компиляторе MyTS позволила повысить гибкость и выразительность кода, обеспечив при
этом полную совместимость с синтаксисом и семантикой TypeScript.
Проведенные оптимизации и тестирование подтвердили эффективность и правильность внедренных решений,
что делает MyTS более мощным и удобным инструментом для разработки высококачественных библиотек и приложений

\newpage
