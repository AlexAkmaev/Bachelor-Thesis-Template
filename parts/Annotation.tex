\begin{abstract}

    \begin{center}
        \large{Поддержка union типов в статическом языке программирования} \\
    \large\textit{Акмаев Алексей Михайлович} \\[1 cm]
    \end{center}

    В данной работе исследуются методы поддержки union типов в статически типизированном языке программирования,
    а также способы генерации и оптимизации байт-кода для этих типов.
    Union типы или объединения позволяют переменной хранить значение одного из нескольких указанных типов, повышая
    безопасность, гибкость и выразительность кода при сохранении преимуществ статической типизации.
    Работа направлена на реализацию union типов в языке, аналогичном TypeScript.
    В ходе исследования выполнены задачи по реализации базовых union типов с необходимыми отношениями подтипирования,
    нормализации объединений, поддержке доступа к полям с одинаковым именем и типом, внедрению литералов в union типы,
    написанию понижающих трансформаций для корректной кодогенерации и оптимизации байт-кода.

    Ключевые слова: union типы; объединения; компилятор; байт-код; нормализация; оптимизация.

    \vfill

\end{abstract}
\newpage